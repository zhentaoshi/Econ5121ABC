%% LyX 2.4.2.1 created this file.  For more info, see https://www.lyx.org/.
%% Do not edit unless you really know what you are doing.
\documentclass[12pt,english]{beamer}
\usepackage{mathpazo}
\renewcommand{\familydefault}{\rmdefault}
\usepackage[T1]{fontenc}
% \usepackage[latin9]{inputenc}
\setcounter{secnumdepth}{3}
\setcounter{tocdepth}{3}
\usepackage[active]{srcltx}
\usepackage{amsthm}
\usepackage{amsmath} 
\usepackage{amssymb}
\usepackage[authoryear]{natbib}
\usepackage{graphicx}

\makeatletter
%%%%%%%%%%%%%%%%%%%%%%%%%%%%%% Textclass specific LaTeX commands.
% this default might be overridden by plain title style
\newcommand\makebeamertitle{\frame{\maketitle}}%
% (ERT) argument for the TOC
\AtBeginDocument{%
  \let\origtableofcontents=\tableofcontents
  \def\tableofcontents{\@ifnextchar[{\origtableofcontents}{\gobbletableofcontents}}
  \def\gobbletableofcontents#1{\origtableofcontents}
}
\theoremstyle{definition}
\newtheorem*{example*}{\protect\examplename}
\theoremstyle{definition}
\newtheorem*{defn*}{\protect\definitionname}
\theoremstyle{plain}
\newtheorem*{thm*}{\protect\theoremname}

%%%%%%%%%%%%%%%%%%%%%%%%%%%%%% User specified LaTeX commands.
\AtBeginDocument{%
   \let\origtableofcontents=\tableofcontents
   \def\tableofcontents{\@ifnextchar[{\origtableofcontents}{\gobbletableofcontents}}
   \def\gobbletableofcontents#1{\origtableofcontents}
 }\usepackage[english]{babel}
\usepackage{babel}

%\usetheme{Boadilla}
\usetheme{Madrid}
\setbeamertemplate{navigation symbols}{}
% \usecolortheme{orchid}
\usecolortheme{spruce}
% \usecolortheme{beaver}

\setbeamercovered{transparent}

\usepackage{colortbl}

\usefonttheme[onlymath]{serif}
%%%%%%%%%%%%%%%%%%%%%%%%

% For tables
\usepackage{multirow}
\usepackage{array}
\usepackage{rotating}
\usepackage{longtable}
\usepackage{float}
\usepackage{booktabs}


% For figures
\usepackage{caption}
\usepackage{subcaption}

\makeatother

\usepackage{babel}
\providecommand{\definitionname}{Definition}
\providecommand{\examplename}{Example}
\providecommand{\theoremname}{Theorem}

% 定义命令 \RomanNum 和 \romanNum
\newcommand{\RomanNum}[1]{\uppercase\expandafter{\romannumeral #1\relax}} % 大写罗马数字
\newcommand{\romanNum}[1]{\romannumeral #1\relax} % 小写罗马数字

\begin{document}
\title[Limited]{Causal Inference}
\author[]{Zhentao Shi}
\date[]{The Chinese University of Hong Kong}

\makebeamertitle

\begin{frame}{Potential Outcome Framework}
    \begin{itemize}
        \item A triple $\left( y_{1i}, y_{0i}, D_{i} \right)$
        \begin{itemize}
            \item $D_{i} \in {0,1}$ is a treatment (from biomedical)
            \item Two potential outcomes $(y_{1i}, y_{0i})$
        \end{itemize}
        
        \item Observed outcome
        \[ y_i = 
        \left\{\begin{matrix}
         y_{1i}, & \text{if}\,\, D_{i} = 1 & \text{treatment group}\\
         y_{0i}, & \text{if}\,\, D_{i} = 0 & \text{control group}
        \end{matrix}\right.
        \]
        Equivalently, $$y_{i} = y_{1i}D_{i} + y_{0i} \left(  1- D_{i} \right)$$
    \end{itemize}
\end{frame}

\begin{frame}{Treatment Effect}
    \begin{itemize}
        \item $\Delta _{i} = y_{1i} - y_{0i}$ is a random variable that varies with individuals\\
        e.g. severity of side effects after people receiving the same vaccine.
        \item $\Delta _{i}$ is unobservable because researcher only observe either $y_{1i}$ or $y_{0i}$, but not both
        \item Heraclitus: ``A man cannot step into the same river twice, because it is not the same river, and he is not same man.''
    \end{itemize}
\end{frame}


\begin{frame}{ATE and ATET}
    \begin{itemize}
        \item Average treatment effect
        \[\text{ATE} = E\left( \Delta_{i}\right)\]
        \item Average treatment effect on the treated
        \[\text{ATET} = E\left( \Delta_{i} \mid D_{i} =1 \right)\]
        \item Control variables $X_{i}$ can be introduced into ATE and ATET
    \end{itemize}
\end{frame}


\begin{frame}{Randomized Controlled Trials (RCT)}
    \begin{itemize}
        \item The golden standard for scientific discovery
        \item Given a sample independently drawn from the same population, randomly assign the treatment to a subsample.
        \item Random assignment implies $$(y_{1i}, y_{0i} ) \bot D_i.$$ The potential outcome is independent of the assignment.
    \end{itemize}
    
\end{frame}


\begin{frame}{ATE Under RCT}
    \begin{itemize}
        \item \textbf{Treatment group} ``$\mathcal{T}$'' ($N_1$ observations) 
        \item \textbf{control group} ``$\mathcal{C}$'' ($N_0 = N - N_1$ observations)
        \begin{align*}
         ATE & = \frac{1}{N_1} \sum_{i \in \mathcal{T}} y_i  - \frac{1}{N_0} \sum_{i \in \mathcal{C}}  y_i \\
             & = \frac{1}{N} \sum_{i=1}^N \left[  \frac{ D_i y_i}{N_1 / N}  -  \frac{(1-D_i) y_i}{N_0 / N} \right]
        \end{align*}
    \end{itemize}
    
\end{frame}


\begin{frame}{RCT in Reality}
    \begin{itemize}
        \item Duflo, India, development
        \item AB test, IT
        \item Both are costly
        \item Not everyone can do it
    \end{itemize}
\end{frame}



\begin{frame}{Conditional ATE and ATET}
    \begin{itemize}
        \item With control variables $X_i = x$, \textbf{conditional ATE} (CATE)
        $$ATE(x) = E[ \Delta_i | X_i = x ]$$
        \item Similar, \textbf{conditional ATET} is defined as 
        $$ATE(x) = E[ \Delta_i | D_i = 1 X_i = x ]$$
        \item Straightforward if $X_i$ is a discrete random variable
    \end{itemize}
\end{frame}



\begin{frame}{Unconfoundedness}
    \begin{itemize}
        
        \item To mimic (unconditional) RCT, it requires \textbf{Conditional Independence}
        \[ \left( y_{1i}, y_{2i} \right) \bot D_{i} \mid X_{i} \]
        \item The above condition is also called \textbf{Unconfoundedness}
        \item $ATET(x) = ATE(x)$ under unconfoundedness.
        \[
        E\left( \Delta_{i} \mid D_{i} =1, X_{i} \right) = E\left( \Delta_{i} \mid X_{i} \right)
        \] \pause


        \item In an \textbf{observational study}, it means ``Once $X_i$ is controlled, the potential outcome is independent of the treatment''
        \item In principle, we should include all \textbf{confounding variables} into $X_i$
        
    \end{itemize}
\end{frame}

\begin{frame}{Overlapping Condition}
    \begin{itemize}
       \item A necessary condition $$0 < \Pr \left( D_{i} = 1\mid X_{i} = x) <1 \right)$$ 

        \item In the subsample $\{X_i = x\}$, define $\mathcal{T}_x$, $\mathcal{C}_x$, $N_x$, $N_{x,1}$, and $N_{x,0}$ are accordingly 
        \begin{align*}
            ATE(x) & = E\left( y_{1i} - y_{0i} \mid X_{i} = x \right)  \\
             &  \overset{\text{C.I.}}{=} E\left( y_{1i} - y_{0i} \mid D_{i}, X_{i} \right)\\
            & = \frac{1}{N_{x,1}} \sum_{i \in \mathcal{T}_x} y_i  - \frac{1}{N_{x,0}} \sum_{i \in \mathcal{C}_x}  y_i  \\
             & = \frac{1}{N_x} \sum_{i=1}^N \left[  \frac{ D_i y_i}{N_{x,1} / N_x}  -  \frac{(1-D_i) y_i}{N_{x,0} / N_x} \right]
        \end{align*}
        
    \end{itemize}
\end{frame}


\begin{frame}{Continuous $X$}
    \begin{itemize}
        
        \item The above analysis is based on $ X \sim \text{discrete random variable}$. 
        \item Assume unconfoundedness and overlapping, and then $$ATE\left(x\right) = ATET \left(x\right)$$ at each point $x$
        \item If $X$ is continuous, one way is to nonparametrically estimate 
        \[ m_{j} \left( x \right) = E\left( y_{ji}  \mid X_{i} = x\right),\, \text{for}\, j \in \left\{ 0,1 \right\}\]
    \end{itemize}    
\end{frame}

\begin{frame}{Propensity Score}
    \begin{itemize}
        
        \item \textbf{Propensity score}: $$P\left(x \right) = \Pr \left( D_{i} = 1 \mid X_{i} = x \right) = E \left( D_{i}  \mid X_{i} = x \right) $$
        \item In the treatment group
        \begin{align*}
        E\left( \frac{D_{i} y_{i}}{P\left( X_{i} \right)} \right) 
         & \overset{\text{LIE}}{=}  E\left( \frac{1}{P\left( X_{i} \right)} E\left( D_{i} y_{1i} \mid X_{i}\right) \right) \\
         & \overset{\text{C.I.}}{=} E \left( \frac{1}{P\left( X_{i} \right)} E \left( D_{i} \mid X_{i} \right) E\left(y_{1i} \mid X_{i} \right)\right) \\
         & \overset{\text{LIE}}{ = } E\left( y_{1i} \right)    
        \end{align*}
        
        \item Similarly,  in the control group $E\left( \frac{\left( 1 - D_{i} \right)y_{i}}{1 - P \left( X_{i} \right)} \right) = E\left( y_{0i} \right)$
        
    \end{itemize}
\end{frame}



\begin{frame}{ATE Under Continuous $X$}
    \begin{itemize}
        \item The (unconditional) ATE is 
        $$ ATE = E\left( \frac{D_{i}y_{i}}{P \left( X_{i} \right)} - \frac{\left( 1 - D_{i} \right) y_{i}}{1 - P\left( X_{i} \right)} \right) $$
        \item Average $ATE(X_i)$ over the support of $X_i$:
        \[ATE = E\left[ ATE\left(X_i\right) \right] = \int ATE \left(X_i\right) \, \mathrm{d}F\left( X_i \right)\]
        % \item Intuition: $E\left( y_{i} \mathbb{I} \left( D_{i} = 1 \right) \right) = E\left( y_{1i} \right) P \left( D_{i} = 1 \right)$ \\need to adjust for the factor $P\left( D_{i} = 1 \right)$.
    \end{itemize}
\end{frame}






\begin{frame}{Regression-based Model}
    \begin{itemize}
        \item Liner potential outcome
        \begin{align*}
        y_{0i} & = \alpha_{0} + X_{i}^{'}\beta_{0} + \varepsilon_{0i}\\
        y_{1i} & = \alpha_{1} + X_{i}^{'}\beta_{1} + \varepsilon_{1i}
        \end{align*}
        The linear assumption makes life easier under continuous $X_i$.
        
        \item The above model implies treatment effect 
        \[\Delta_{i} = \left( \alpha_{1} - \alpha_{1} \right) + X_{i}^{'} \left( \beta_{1} - \beta_{0} \right) + \left( \varepsilon_{1i} - \varepsilon_{0i}\right)\]
        \item By construction $E\left( \varepsilon_{1i} \mid X_{i} \right) = \left( \varepsilon_{0i} \mid X_{i} \right) = 0$, and thus
        
        \[ATE(X_i) =  E\left( \Delta_{i} \mid X_{i} \right) =  \left( \alpha_{1} - \alpha_{1} \right) + X_{i}^{'} \left( \beta_{1} - \beta_{0} \right)\]
        
    \end{itemize}
\end{frame}

\begin{frame}{Selection Bias}
\begin{align*}
   ATET(X_i) & = E\left( \Delta_{i} \mid D_{i} =1, X_{i} \right) \\
    & = ATE(X_i) + E\left( \varepsilon_{1i} - \varepsilon_{0i} \mid D_{i} =1, X_{i} \right) 
\end{align*}

    \begin{itemize}
         
        \item Under unconfoundedness, $ATE(X_i) = ATET(X_i)$
        \item Otherwise, \textbf{selection bias} if $$E\left( \varepsilon_{1i} - \varepsilon_{0i} \mid D_{i} =1, X_{i}  \right) \ne 0$$
        The individual knows $\varepsilon_{1i} - \varepsilon_{0i}$, and he elects to the treatment group because of that.
        \item The fact that he joins the treatment group is a calculated rational decision, not an indifferent random assignment.
        \item e.g.~Children of a civil servant are more likely to become a civil servant (but if we fail to control family background)
        
    \end{itemize}
\end{frame}

% I would like to talk about LATE.
% However, LATE is not covered by the textbook

\begin{frame}{Regression Discontinuity Design}
    \begin{itemize}
        \item Quasi experiment: A jump point due to some ad hoc policy cutoff
        \item Unconfoundedness is naturally satisfied
        \item e.g.~The value-added of college
        \item Caveat: valid only for the subpopulation around the cutoff
        \item Suppose the cutoff is $c$
        \item $y_{0i} = \mu_0 + \beta_0 (x_i - c) + \varepsilon_{0i}$
        \item $y_{1i} = \mu_1 + \beta_1 (x_i - c) + \varepsilon_{1i}$        
    \end{itemize}
\end{frame}




\begin{frame}{Example: Air Pollution in Huai River}
\begin{itemize}
    \item \textcolor{red}{insert a graph}
    \item Example: Chen, Ebenstein, Greenstone, Li (2013, PNAS) 
\end{itemize}

    
\end{frame}

    



\begin{frame}{Event Study}
\begin{itemize}
    \item A time series topic, but very similar to treatment
    \item The same individual observed over time $t = 1,2,\ldots,T$
    \item An event happens at time $t = T_1$
    \begin{itemize}
        \item Before event (control group)
        \item After event (treatment group)
        \item $D_t = \mathbb{I}(t \geq T_1) $
    \end{itemize}
    \item Regression
    $$y_t = \alpha + \beta D_t + X_t' \gamma + \varepsilon_t$$
    \item Key assumption
    $$E[\varepsilon_t | X_t, D_t ] = 0 \mbox{ for all } t = 1,2\ldots,T $$
    
\end{itemize}
\end{frame}

\begin{frame}{Difference in Difference}
    \begin{itemize}
        \item Before and after
        \item Treatment and control
        
    \end{itemize}
\end{frame}



\begin{frame}{Pooled Cross Sections}
    \begin{itemize}
        \item Cross-sectional datasets collected at different time points  
        \item Group-specific intercept  
        \item Analysis of time trends  
        \item Example: Time trend of migrant workers' wages in Shenzhen
    \end{itemize}
\end{frame}

\begin{frame}{Difference-in-Difference (DID) Method}
    \begin{itemize}
        \item Isolate the treatment effect  
        \item \emph{difference-in-difference} (DID)  
        \item Example: Effect of the HKU Station opening on rental prices in the nearby area, with HKU as the treatment group and CUHK as the control group.
    \end{itemize}
\end{frame}



\end{document}